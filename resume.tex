\documentclass[]{iplresume} %Evil options: narrowline, wideline

%Font Info
\usepackage[T1]{fontenc}

%Document Info
\title{機械学習アルゴリズムの分散グラフ処理フレームワークによる記述}
\author{Rathore Amogh}
\stdnum{1311216}
\office{岩崎研究室}
\lhead{電気通信大学 情報理工学部 情報・通信工学科 平成28年度卒業論文中間発表}
\rhead{平成28年09月30日(金曜日)}

\begin{document}
\maketitle
\section{背景}
グラフはとてもフレキシブルな表現である。いろいろな分野のデータがグラフで表現できて、様々な問題が形式化できる。今はビッグデータの時代なので、量が大きいデータをグラフで表現すると非常にも大きいグラフになることが普段である。そのような大きいグラフは一つの計算機で処理するのは無理である。なぜなら、一つの計算機はメモリーが限られているし、プロセサが一個なので、処理時間も長いからである。だから、グラフを分散して処理する必要がある。

分散グラフ処理はフレームワークがいくつか知られている。例えば、Apache GiraphやPregel+などのフレームワークが一番有名である。

近頃は、ビッグデータと同様に機械学習の分野も広がれてきた。機械学習は人間が自然に行っている学習と同様の機能をコンピュータで実現する技術のことである。基本的には、機械学習がデータにパターンを見つけてデータについて学習を行うことである。ここもデータが大きいと処理時間が長くなるので、並列的に処理を行う必要がある。

\section{目的と方針}
本研究では、機械学習のアルゴリズムを分散グラフ処理フレームワーク

\subsection{レジュメの紙面が}
\subsubsection{余る}
\paragraph{そんな君の}学位論文は内容がない.
\section{まとめ}
ぞうの卵はおいしいぞう.
ぞうの卵はおいしいぞう.
ぞうの卵はおいしいぞう.
ぞうの卵はおいしいぞう.
ぞうの卵はおいしいぞう.
ぞうの卵はおいしいぞう.
ぞうの卵はおいしいぞう.
ぞうの卵はおいしいぞう.
ぞうの卵はおいしいぞう.
ぞうの卵はおいしいぞう.

\appendix
\section{科研費\LaTeX}

\end{document}
